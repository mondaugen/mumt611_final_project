\documentclass[72pt]{article}
\usepackage[a4paper,margin=1in,landscape]{geometry}
\usepackage{mathtools}
\DeclareMathSizes{72}{72}{72}{72}
\begin{document}

\begin{Huge}
\title{Monaural Singing Voice Separation Using Non-Negative Matrix 
Factorization}
\date{April 29th, 2015}
\author{Nicholas Esterer}
\maketitle

\newpage

\section*{Motivation: Separating the Singing Voice from Instrumental
Accompaniment}
Applications:
\begin{itemize}
    \item Melody extraction
    \item Song recognition    
    \item Singer identification
\end{itemize}

\newpage

\section*{Features of the Singing Voice}
\begin{itemize}
    \item Compared to harmonic instrumental accompaniment, the singing voice
            exhibits a spectrum whose frames vary over time whereas instrumental
            accompaniment remains static.
    \item Compared to percussive instrumental accompaniment, the singing voice
            spectrum does not vary as greatly over time as brief percussive
            impulses
\end{itemize}

\newpage

\section*{Why?}
\begin{itemize}
    \item In Marin and McAdams (1991), studies showed that modulating the
            frequency and vowel of a synthesized tone increased its prominence
            among unmodulated vowels.
    \item This suggest that singers modulate frequency and amplitude of the
voice to have prominence over other harmonic instruments.
\end{itemize}

\newpage

\section*{Characteristics of Singing Voice Spectrograms}
\begin{itemize}
    \item Because the singing voice exhibits some frequency and amplitude
    modulation, taking a long window spectrogram of the voice will show
    excitation in different bins of adjacent frames, while a spectrogram of the
    more static harmonic accompaniment will show adjacent frames sharing
    prominent bins.
\end{itemize}

\newpage

% Insert long window slides of voice and instrument

\begin{itemize}
    \item Because the singing voice exhibits less frequency and amplitude
    modulation than percussive instruments, taking a short window spectrogram of
    the singing voice will show excitation of the same bin in adjacent frames
    whereas the spectrogram of a percussive sound will not share prominent bins
    among adjacent frames.
\end{itemize}

\newpage

% Insert short window slides of voice and instrument

\section*{Technique}
\begin{itemize}
    \item These observations motivate a technique that removes those elements of
the spectrogram that are not deemed to stem from the voice:
    \begin{itemize}
        \item Remove spectra whose prominent bins are non-varying on a
long-window spectrogram.
        \item Remove spectra whose prominent bins are varying on a short-window
spectrogram
    \end{itemize}
\end{itemize}

\newpage
\begin{itemize}
    \item Of course, the singing voice and the instrumental accompaniment will often be
    simultaneous in the recording and their respective spectra also simultaneous.
    \item We would like to separate these spectra.
    \item Hennequin et al. (2011), among others, demonstrated that non-negative
matrix factorization (NMF) is good at separating spectrograms into a set of basis
vectors \( \mathbf{B} \) and their time-varying gains \( \mathbf{G} \).
    \item We use NMF on both the long- and short-window spectrograms to compute
basis spectra.
    \item We then identify spectra stemming from harmonic instruments and
percussive instruments, respectively.
\end{itemize}

\newpage

The following calculation is used to measure the amount of spectral
discontinuity in the \( j^{th} \) column of \( \mathbf{B} \) (a basis vector) in
order to identify spectra stemming from harmonic instruments

\[ 
d_s(\mathbf{B}^j) =
\frac{\sum\limits_{k=2}^{K}(\mathbf{B}_{k,j}-\mathbf{B}_{k-1,j})^2}{
    \sum\limits_{k=1}^{K}(\mathbf{B}_{k,j}^{2})}
\]

where \( K \) is the number of bins in the spectrogram.

\newpage

Similarly, the following calculation is used to measure the amount of
discontinuity in the activation gains in the \( j^{th} \) column of \(
\mathbf{G} \) in order to identify spectra stemming from percussive instruments

\[ 
d_s(\mathbf{G}^j) =
\frac{\sum\limits_{t=2}^{T}(\mathbf{G}_{j,t}-\mathbf{G}_{j,t-1})^2}{
    \sum\limits_{t=1}^{T}(\mathbf{G}_{j,t}^{2})}
\]

where \( T \) is the number of frames in the spectrogram.

\newpage

\section*{The Computation of Spectral Bases}
Given a spectrogram \( \mathbf{X} \) of \( K \) bins and \( T \) frames, NMF
will find an approximate factorization

\[ \mathbf{X} \approx \mathbf{B}\mathbf{G} \]

where \( \mathbf{B} \) is a matrix of \( K \) rows and \( J \) columns and \(
\mathbf{G} \) is a matrix of \( J \) rows and \( T \) columns.

\newpage

When using the Kullback-Leibler divergence and interpreting it as a
\(\beta\)-divergence where \(\beta = 1 \), according to Hennequin et al.
(2011b), the divergence is non-increasing when using the following update rules:

\[ \mathbf{B} \gets
\mathbf{B}.\frac{\frac{\mathbf{X}}{\mathbf{B}\mathbf{G}}\mathbf{G}^T}
    {\mathbf{1}\mathbf{G}^T}
\]

\[ \mathbf{G} \gets
\mathbf{G}.\frac{\mathbf{B}^T\frac{\mathbf{X}}{\mathbf{B}\mathbf{G}}}
    {\mathbf{B}^{T}\mathbf{1}}
\]

Where \( . \) indicates element-wise multiply and \( \frac{}{} \) indicates
element-wise division.



\end{Huge}




\end{document}
